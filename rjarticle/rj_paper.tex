% !TeX root = RJwrapper.tex
\title{corVis: An R Package for Visualising Associations and Conditional
Associations}
\author{by Amit Chinwan and Catherine Hurley}

\maketitle

\abstract{%
Correlation matrix displays are important tools to explore multivariate
datasets. These displays with other measures of association can
summarize interesting patterns to an analyst and assist them in framing
questions while performing exploratory data analysis. In this paper, we
present new visualisation techniques to visualise association between
all the variable pairs in a dataset in a single plot, which is something
existing displays lack. Also, we propse new methods to visualise
relationship among variable pairs using conditioning. We use different
layouts like matrix or linear for our displays. We use seriation in our
displays which helps in highlighting interesting patterns easily. The R
package \texttt{corVis} provides an implementation.
}

\hypertarget{section-1-introduction}{%
\section{Section 1: Introduction}\label{section-1-introduction}}

Correlation matrix display is a popular tool to visually explore
correlations among variables while performing Exploratory Data Analysis
(EDA) on a multivariate dataset. Popularized by
\citet{friendly2002corrgrams} as corrgram, these displays are produced
by first calculating the correlation among the variables and then
plotting these calculated values in a matrix display. With effective
ordering techniques, these displays quickly highlight variables which
are highly correlated and an analyst interested in building a predictive
model could use these displays to remove correlated variables and avoid
multicollinearity.

The correlation displays are generally used with one of the Pearson's,
Spearman's or Kendall's correlation coefficient and are therefore
limited to quantitative variables. An analyst can use one-hot encoding
of the qualitative variables in order to use these displays but will
need to deal with the high dimensions as a result of the encoding. In
addition to the dimensionality problem, it is not easy to assess the
overall correlation when using the one-hot encoding. The existing
methods to quickly explore association among qualitative variables in a
dataset include using proportions or counts with different graphical
displays like boxplots or barplots. Using association measures for
qualitative pairs similar to correlation for quantitative pairs will
help in summarizing the relationship, which then can be displayed like
the correlation displays.

Tukey and Tukey introduced scagnostics which are measures for
scatterplots \citep{tukey1985computer}. Along with scagnostics, they
proposed a scagnostics scatterplot matrix which is a visual display to
explore and compare these measures for all the variable pairs in a
dataset. By comparing multiple measures at once, the unusual variable
pairs could be identified and looked at in more detail. In a similar
manner, a display comparing association measures will help in finding
interesting variable pairs. Many association measures have been proposed
to summarize different types of relationships. The most commonly used
measure is Pearson's correlation coefficient which captures any linear
trend present between the variables. Other popular measures include
Kendall's or Spearman's rank correlation coefficient which are
non-parametric measures and looks for monotonic relationship. Distance
correlation \citep{szekely2007measuring} is an important measure useful
in exploring non-linear relationships. The information theory measure
maximal information coefficient (MIC) \citep{reshef2011detecting} is
capable of summarizing complex relationships. With effective displaying
techniques, the multiple measures of association provide a comparison
tool that assist an analyst to reveal structure present in the data.

Small multiples (or Trellis display) is a simple yet powerful approach
to compare partitions of data and understand multidimensional datasets
\citep{tufte1986thevisual}. The display is produced by splitting the
data into groups by a conditioning variable and then plotting the data
for each group. Such displays allow analysts to quickly infer about the
impact of the conditioning variable. A similar idea applied to displays
of association measures (correlation plot) will help uncover underlying
patterns in the data. One such pattern is Simpson's paradox which can be
detected by comparing Pearson's correlation for data at overall level
versus individual levels of the conditioning variable.

In this paper, we propose extensions of the correlation plot and new
visualizations which look at variables of mixed type, multiple
association measures and conditional associations. These displays are
implemented in the R package \CRANpkg{corVis}. The next section provides
a review of existing packages which deal with correlation displays and a
quick background on association measures and the packages used for
calculating them. Then we describe our approach to calculate the
association measures, followed by visualizations of associations and
conditional associations. We conclude with a summary and future work.

\hypertarget{section-2-background}{%
\section{Section 2: Background}\label{section-2-background}}

In this section we provide a brief review of existing packages used for
correlation displays and association measure calculation.

\hypertarget{section-2.1-literature-review-on-correlation-displays}{%
\subsection{Section 2.1: Literature Review on Correlation
Displays}\label{section-2.1-literature-review-on-correlation-displays}}

According to \citet{hills1969looking}, ``the first and sometimes only
impression gained by looking at a large correlation matrix is its
largeness''. To overcome this, \citet{murdoch1996graphical} proposed a
display for large correlation matrices which uses a matrix layout of
ellipses where the parameters of the ellipses are scaled to the
correlation values. \citet{friendly2002corrgrams} expanded on this idea
by rendering correlation values as shaded squares, bars, ellipses, or
circular `pac-man' symbols. The variables in the matrix displays were
optionally ordered using the angular ordering of the first two eigen
vectors of the correlation matrix. The ordering places highly-correlated
pairs of variables nearby, making it easier to quickly identify groups
of variables with high mutual correlation.

Nowadays, there are many R packages devoted to correlation
visualisation. Table \ref{tab:corrdisplay-packages} provides a summary,
listing the displays offered, and whether these extend to factor
variables or mixed numeric-factor pairs.

The R package \CRANpkg{corrplot} \citep{corrplot2021} provides an
implementation of the methods in \citet{friendly2002corrgrams}. The
package \CRANpkg{corrr} \citep{corrr2020} organises correlations as tidy
data, so leveraging the data manipulation and visualisation tools of the
\CRANpkg{tidyverse} \citep{tidyverse}. In addition to various matrix
displays, the package offers network displays where line-thickness
encodes correlation magnitude, with a filtering option to discard
low-correlation edges.

The package \CRANpkg{corrgrapher} \citep{corrgrapher} uses a network
plot for exploring correlations, where the nodes close to each other
have high correlation magnitude, edge thickness encodes the absolute
correlation value and edge color indicates the sign of correlation. The
package also handles mixed type variables by using association measures
obtained as transformations of \(p\)-values obtained from Pearson's
correlation test in the case of two numeric variables, Kruskal's test
for numerical and factor variables, and a chi-squared test for two
categorical variables.

The package \CRANpkg{linkspotter} \citep{linkspotter} offers a variety
of association measures (distance correlation, MIC, maximum normalized
mutual information) in addition to correlation, where the measure used
depends on whether the variables are both numerical, categorical or
mixed. The results are visualized in a network plot, which may be
packaged into an interactive shiny application.

Our own package \CRANpkg{corVis} offers a variety of displays, and has
new features not available elsewhere, in particular simultaneous display
of multiple association measures, and association displays stratified by
levels of a grouping variable. This will be described in the following
sections.

There have been other extensions to correlation displays which are
useful when dealing with high dimensional datasets.
\citet{hills1969looking} proposed a QQ plot of the \(z\)-transform of
the entries of the correlation matrix to discover correlation
coefficients too large to come from a normal distribution with mean
zero. \citet{buja2016visualization} proposed Association Navigator which
is an interactive visualization tool for large correlation matrices with
upto 2000 variables. The R package \CRANpkg{scorrplot} \citep{scorr}
produces an interactive scatterplot for exploring pairwise correlations
in a large dataset by projecting variables as points and encoding the
correlations as space between these points. The package provides a
functionality to update variable of interest which creates tour of the
correlation space between different projections of the data.

The R package \CRANpkg{correlationfunnel} offers a novel display which
assists in feature selection in a setting with a single response and
many predictor variables. All numeric variables including the response
are binned. All (now categorical) variables in the resulting dataset are
one-hot encoded and Pearson's correlation calculated with the response
categories. The correlations are visualised in a dot-plot display, where
predictors are ordered by maximum correlation magnitude. Correlations
between one-hot encoded variables are challenging to interpret,
especially as the number of levels increase. In corVis we offer a
similar dot-plot display, but showing multiple correlation or
association measures, or alternatively measures stratified by a grouping
variable.

\begin{Schunk}
\begin{table}

\caption{\label{tab:corrdisplay-packages}List of the R packages dealing with correlation or correlation displays with information on whether the plots display multiple measures, conditional display of measures and mixed variables in a single plot}
\centering
\begin{tabular}[t]{lll}
\toprule
Package & Display & MixedVariables\\
\midrule
corrplot & heatmap & \\
corrr & heatmap/network & \\
corrgrapher & network & \\
linkspotter & network & Yes\\
correlation & heatmap/network & \\
\addlinespace
corVis & heatmap/matrix/linear & Yes\\
\bottomrule
\end{tabular}
\end{table}

\end{Schunk}

\hypertarget{section-2.2-literature-review-on-association-measures}{%
\subsection{Section 2.2: Literature Review on Association
Measures}\label{section-2.2-literature-review-on-association-measures}}

An association measure can be defined as a numerical summary quantifying
the relationship between two or more variables. For example, Pearson's
correlation coefficient summarizes the strength and direction of the
linear relationship present between two numeric variables and is in the
range \([-1,1]\). Kendall's or Spearman's rank correlation coefficient
are other popular measures which assess montonic relationship among two
numeric variables and are in the range \([-1,1]\). As these measures are
limited to linear or monotonic relationships, there's a need to use
association measures which are able to capture complex relationships. In
addition to association measures for numeric variables, association
measures for ordinal, nominal and mixed variable pairs are useful in
exploring a multivariate dataset. We now give an overview of available
association measures.

For a pair of numeric variables, various measures of association have
been proposed in literature. The distance correlation coefficient
\citep{szekely2007measuring} is an association measure which looks for
the non-linear association between two numeric variables and summarizes
it in \([0,1]\). Similarly, MIC \citep{reshef2011detecting} is capable
of summarizing non-linear as well as periodic relationships between
numeric variables and is in range \([0,1]\).

\citet{agresti2010analysis} provides an overview of the association
measures which are used for exploring association between ordinal
variables. Kendall's tau-b \citep{kendall1945treatment} is an
association measure useful in summarizing the relationship between two
ordinal variables in the range \([-1,1]\). It is a relatively stable
measure than Goodman and Kruskal's gamma with respect to the changes in
categories of any variable i.e.~if two categories are merged to make a
single category. The polychoric correlation \citep{olsson1979maximum}
measures the correlation between two ordinal variables by assuming two
normally distributed latent variables for a contingency table of two
ordinal variables and summarizes the association in \([-1,1]\).

The association measures for the case of nominal pair of variables
should be invariant to the order in which the categories appear.
Pearson's contingency coefficient uses the \({\chi}^2\) value from the
Pearson's \({\chi}^2\) test for independence and is a useful measure to
summarize the association between two nominal variables in \([0,1]\).
Another measure for nominal variable pair is the Uncertainty coefficient
\citep{theil1970estimation} measuring the proportion of uncertainty in
one variable which is explained by the other variable.

\hypertarget{section-3-introducing-corvis}{%
\section{Section 3: Introducing
corVis}\label{section-3-introducing-corvis}}

\CRANpkg{corVis} is an R package which calculates measures of
association for every variable pair in a dataset and helps in
visualising these associations in different ways. Most of the existing
correlation displays are limited to numeric pairs of variables. This
package extends these displays to every variable pair.

The package offers new visualisation technique such as display with
multiple measures for the association for every variable pair in the
dataset. We also introduce conditional association displays which are
useful in uncovering conditional structure present in the data and are
produced by splitting the data by a partitioning variable, calculating
association for the variable pairs at every level of partitioning
variable and then plotting them. These displays are efficient for
discovering differences among the groups in the data.

Efficient seriation techniques have been included to order and highlight
variables with high value for an association measure. These ordered
association and conditional association displays are useful in finding
interesting patterns, such as Simpson's Paradox. These new displays also
help an analyst to quickly discover any unusual variable pair(s) in the
dataset.

Table \ref{tab:function-corVis} provides a list of the functions
available in the package which are useful for calculating association
measures among variable pairs and visualising these associations using
novel displays. The function \texttt{calc\_assoc} calculates the
association measures for variable pairs in a dataset and is described in
detail in Section 4. The remaining functions in table
\ref{tab:function-corVis} are useful for displaying pairwise association
and their usage has been illustrated in Section 5.

\begin{Schunk}
\begin{table}

\caption{\label{tab:function-corVis}List of the available functions in corVis package with input arguments and outputs}
\centering
\begin{tabular}[t]{>{}lll}
\toprule
Function & Usage & Input\\
\midrule
\textbf{calc\_assoc} & Calculation & Dataframe\\
\textbf{} &  & Types of association measures\\
\textbf{} &  & NA handler\\
\textbf{} &  & Name of the grouping \vphantom{1} variable\\
\textbf{} &  & Include overall value or not\\
\addlinespace
\textbf{association\_heatmap} & Visualization & Measures dataframe for lower/upper triangle\\
\textbf{} &  & Variable \vphantom{3} order\\
\textbf{} &  & Limits of the \vphantom{3} scale\\
\textbf{pairwise\_2d\_plot} & Visualization & Measures dataframe for lower/upper triangle\\
\textbf{} &  & Name of the grouping variable\\
\addlinespace
\textbf{} &  & fill \vphantom{1} variable\\
\textbf{} &  & Variable \vphantom{2} order\\
\textbf{} &  & Limits of the \vphantom{2} scale\\
\textbf{pairwise\_1d\_compare} & Visualization & Dataframe\\
\textbf{} &  & Measures for the display\\
\addlinespace
\textbf{} &  & Variable \vphantom{1} order\\
\textbf{} &  & Limits of the \vphantom{1} scale\\
\textbf{pairwise\_1d\_plot} & Visualization & Measures dataframe\\
\textbf{} &  & Name of the grouping variable for display\\
\textbf{} &  & fill variable\\
\addlinespace
\textbf{} &  & Variable order\\
\textbf{} &  & Limits of the scale\\
\bottomrule
\end{tabular}
\end{table}

\end{Schunk}

\hypertarget{section-4-corvis-calculating-association}{%
\section{Section 4: corVis: Calculating
Association}\label{section-4-corvis-calculating-association}}

This section describes the calculation of association measures in our
package \CRANpkg{corVis}. The package provides a collection of various
measures of association which quantifies the relationship between two
variables. The association measures available in the package are not
limited to numeric variables and are used with nominal, ordinal and
mixed variable pairs as well. Table \ref{tab:association-measures} lists
different functions provided in the package to calculate varoius
measures of association. The \texttt{funName} column represents the
function name used to calculate measure(s) of associations in this
package. The \texttt{typeX} and \texttt{typeY} columns provide the
information on types of variables which can be used with the
corresponding functions. The \texttt{X} or \texttt{Y} variable is one of
the numeric, nominal, ordinal or any type. The \texttt{from} column
corresponds to the package functions used to calculate the association
measures by the function under \texttt{funName}. The \texttt{symmetric}
column represents if the measure is symmetric i.e.~if the value of
measure is same regardless of the order of variables. The last column
provides the range of values for these measures. The function
\texttt{tbl\_easy} calculates association measures available in the R
package \CRANpkg{correlation} and is suitable for different variable
types. The functions in Table \ref{tab:association-measures} with
\texttt{corVis} entries under \texttt{from} column calculate the
association measures which have been implemented in this package.

For numeric pairs of variables, this package provides a range of
association measures. The popular correlation coefficients like
Pearson's or Spearman's or Kendall's are calculated using
\texttt{tbl\_cor} function. The measures such as distance correlation or
MIC which assess more complex relationship are calculated using
\texttt{tbl\_dcor} or \texttt{tbl\_mine} respectively. The association
measures available in the package for the ordinal pairs of variables are
polychoric correlation and Kendall's coefficients which are calculated
using \texttt{tbl\_polycor} or \texttt{tbl\_tau} respectively. For
nominal pairs of variables, the functions like \texttt{tbl\_gkTau},
\texttt{tbl\_gkGamma}, \texttt{tbl\_uncertainty}, \texttt{tbl\_chi},
\texttt{tbl\_cancor} are used for exploring association among the
variables.

The function \texttt{tbl\_cancor} calculates a measure of association
based on canonical correlations for mixed pairs of variables. Nominal
variables are converted into sets of dummy variables, which are then
assigned scored to find the maximal correlation. For two numeric
variables this measure is identical to absolute correlation, for two
factors the correlation is identical to that obtained from
correspondence analysis.

The functions listed in \ref{tab:association-measures} for calculating
association measures provide a functionality for handling missing value
or \texttt{NA} in the dataset. Each of these functions either have a
\texttt{handle.na} argument or have package functions which
automatically uses pairwise complete observations for taking care of
missing values present in the data.

\begin{Schunk}
\begin{table}

\caption{\label{tab:association-measures}List of the functions available in the package for calculating different association measures along with the packages used for calculation.}
\centering
\begin{tabular}[t]{llllll}
\toprule
Function & X & Y & from & symmetric & range\\
\midrule
tbl\_cor & numerical & numerical & stats::cor & Y & {}[-1,1]\\
tbl\_dcor & numerical & numerical & energy::dcor2d & Y & {}[0,1]\\
tbl\_mine & numerical & numerical & minerva::mine & Y & {}[0,1]\\
tbl\_polycor & ordinal & ordinal & polycor::polychor & Y & {}[-1,1]\\
tbl\_tau & ordinal & ordinal & DescTools::KendalTauA,B,C,W & Y & {}[-1,1]\\
\addlinespace
tbl\_gkGamma & ordinal & ordinal & DescTools::GoodmanKruskalGamma & Y & {}[0,1]\\
tbl\_gkTau & nominal & nominal & DescTools::GoodmanKruskalTau & N & {}[0,1]\\
tbl\_uncertainty & nominal & nominal & DescTools::UncertCoef & Y & {}[0,1]\\
tbl\_chi & nominal & nominal & DescTools::ContCoef & Y & {}[0,1]\\
tbl\_cancor & nominal/numerical & nominal/numerical & corVis & Y & {}[0,1]\\
\addlinespace
tbl\_nmi & any & any & corVis & Y & {}[0,1]\\
tbl\_easy & any & any & correlation::correlation & Y & {}[-1,1]\\
\bottomrule
\end{tabular}
\end{table}

\end{Schunk}

\hypertarget{calculating-association-for-a-single-type-of-variable-pairs}{%
\subsection{Calculating association for a single type of variable
pairs}\label{calculating-association-for-a-single-type-of-variable-pairs}}

We introduce a method which creates a tibble structure for the variable
pairs in a dataset along with calculated association measure. The
package contains various functions (shown in Table
\ref{tab:association-measures}) for different association measures in
the form \texttt{tbl\_*} to calculate them. For example, someone
interested in calculating distance correlation for numeric pair of
variables in a dataset is done by using \texttt{tbl\_dcor} .

\begin{Schunk}
\begin{Sinput}
ckd <- RWeka::read.arff("../data/chronic_kidney_disease.arff")
df <- ckd
distance <- tbl_dcor(df)
distance
\end{Sinput}
\begin{Soutput}
#> # A tibble: 55 x 4
#>    x     y     measure measure_type
#>    <chr> <chr>   <dbl> <chr>       
#>  1 bp    age     0.189 dcor        
#>  2 bgr   age     0.303 dcor        
#>  3 bu    age     0.249 dcor        
#>  4 sc    age     0.242 dcor        
#>  5 sod   age     0.154 dcor        
#>  6 pot   age     0.119 dcor        
#>  7 hemo  age     0.254 dcor        
#>  8 pcv   age     0.296 dcor        
#>  9 wbcc  age     0.173 dcor        
#> 10 rbcc  age     0.307 dcor        
#> # ... with 45 more rows
\end{Soutput}
\end{Schunk}

The tibble output for the functions mentioned in Table
\ref{tab:association-measures} has the following structure:

\begin{itemize}
\tightlist
\item
  \texttt{x} and \texttt{y} representing a pair of variables
\item
  \texttt{measure} representing the calculated value for association
  measure
\item
  \texttt{measure\_type} representing the association measure calculated
  for \texttt{x} and \texttt{y} pair.
\end{itemize}

\hypertarget{calculating-association-measures-for-whole-dataset}{%
\subsection{Calculating association measures for whole
dataset}\label{calculating-association-measures-for-whole-dataset}}

\texttt{calc\_assoc} is used to calculate association measures for all
the variable pairs in the dataset at once in a tibble structure. The
variable pairs in the output are unique pairs and a subset of all the
pairs of variables in a dataset where \texttt{x} \(\neq\) \texttt{y}.
Because of the tidy structure of the output, the data manipulation and
visualisation tools of \CRANpkg{tidyverse} \citep{tidyverse} are
applicable to and are useful for further exploration of pairwise
associations. In addition to tibble structure, the output also has
\texttt{pairwise} and \texttt{data.frame} class which are important
class attributes for producing visual summaries in this package.

The function \texttt{calc\_assoc} has a \texttt{types} argument which is
basically a tibble of the association measure to be calculated for
different variable pairs. The default tibble of measures is
\texttt{default\_assoc()} which calculates Pearson's correlation if both
the variables are numeric, Kendall's tau-b if both the variables are
ordinal, canonical correlation if one is factor and other is numeric and
canonical correlation for the rest of the variable pairs.

\begin{Schunk}
\begin{Sinput}
default_measures <- default_assoc()
default_measures
\end{Sinput}
\begin{Soutput}
#> # A tibble: 4 x 4
#>   funName    typeX   typeY   argList
#>   <chr>      <chr>   <chr>   <list> 
#> 1 tbl_cor    numeric numeric <NULL> 
#> 2 tbl_tau    ordered ordered <NULL> 
#> 3 tbl_cancor factor  numeric <NULL> 
#> 4 tbl_cancor other   other   <NULL>
\end{Soutput}
\begin{Sinput}
ckd_assoc <- calc_assoc(df,types = default_assoc())
ckd_assoc
\end{Sinput}
\begin{Soutput}
#> # A tibble: 300 x 4
#>    x     y     measure measure_type
#>    <chr> <chr>   <dbl> <chr>       
#>  1 bp    age    0.159  pearson     
#>  2 sg    age    0.199  cancor      
#>  3 al    age    0.235  cancor      
#>  4 su    age    0.287  cancor      
#>  5 rbc   age    0.0800 cancor      
#>  6 pc    age    0.151  cancor      
#>  7 pcc   age    0.158  cancor      
#>  8 ba    age    0.0422 cancor      
#>  9 bgr   age    0.245  pearson     
#> 10 bu    age    0.197  pearson     
#> # ... with 290 more rows
\end{Soutput}
\begin{Sinput}
class(ckd_assoc)
\end{Sinput}
\begin{Soutput}
#> [1] "pairwise"   "tbl_df"     "tbl"        "data.frame"
\end{Soutput}
\end{Schunk}

The default tibble of measures is updated using the
\texttt{update\_assoc} function which has arguments for updating the
\texttt{tbl\_*} functions to calculate association measures depending on
the type variable pair in the dataset and a method for \texttt{tbl\_*}
functions which calculates more than one measure. The
\texttt{update\_assoc} function has an argument \texttt{default} which
has the \texttt{default\_assoc()} tibble as its default value and is
useful when \texttt{tbl\_*} functions need to be updated for a few types
of variable pairs.

\begin{Schunk}
\begin{Sinput}
updated_assoc <- update_assoc(default=default_assoc(),
                              num_pair = "tbl_cor",
                              num_pair_argList = "spearman",
                              mixed_pair = "tbl_cancor",
                              other_pair = "tbl_nmi")
updated_assoc
\end{Sinput}
\begin{Soutput}
#> # A tibble: 4 x 4
#>   funName    typeX   typeY   argList  
#>   <chr>      <chr>   <chr>   <list>   
#> 1 tbl_cor    numeric numeric <chr [1]>
#> 2 tbl_tau    ordered ordered <NULL>   
#> 3 tbl_cancor factor  numeric <NULL>   
#> 4 tbl_nmi    other   other   <NULL>
\end{Soutput}
\end{Schunk}

\texttt{calc\_assoc} also has a \texttt{handle.na} argument for handling
the \texttt{NA} or missing values which is fed into the \texttt{tbl\_*}
functions used with the \texttt{types} argument for different types of
variable pairs. The default value is set to \texttt{TRUE} for using
pairwise complete observations for calculating a measure of association
between two variables.

If a user is interested in calculating multiple association measures for
a type of variable pair, it can be done by using the
\texttt{calc\_assoc} and \texttt{update\_assoc} together for calculating
different association measures and then merging the output tibbles.

\begin{Schunk}
\begin{Sinput}
updated_ckd_assoc <- calc_assoc(df, types = updated_assoc)
updated_ckd_assoc
\end{Sinput}
\begin{Soutput}
#> # A tibble: 300 x 4
#>    x     y     measure measure_type
#>    <chr> <chr>   <dbl> <chr>       
#>  1 bp    age    0.123  spearman    
#>  2 sg    age    0.199  cancor      
#>  3 al    age    0.235  cancor      
#>  4 su    age    0.287  cancor      
#>  5 rbc   age    0.0800 cancor      
#>  6 pc    age    0.151  cancor      
#>  7 pcc   age    0.158  cancor      
#>  8 ba    age    0.0422 cancor      
#>  9 bgr   age    0.299  spearman    
#> 10 bu    age    0.309  spearman    
#> # ... with 290 more rows
\end{Soutput}
\end{Schunk}

\hypertarget{calculating-conditional-association}{%
\subsection{Calculating conditional
association}\label{calculating-conditional-association}}

\texttt{calc\_assoc} is also used to calculate association measures for
all the variable pairs at different levels of a categorical variable.
This helps in exploring the conditional associations and find out the
differences between the groups of the conditioning variable. The
function has a \texttt{by} argument which is used as the grouping
variable and needs to be categorical.

\begin{Schunk}
\begin{Sinput}
ckd_assoc_by <- calc_assoc_by(df,by = "cad")
ckd_assoc_by
\end{Sinput}
\begin{Soutput}
#> # A tibble: 1,104 x 5
#>    x     y     measure measure_type by   
#>    <chr> <chr>   <dbl> <chr>        <fct>
#>  1 bp    age    -0.307 pearson      yes  
#>  2 sg    age     0.421 cancor       yes  
#>  3 al    age     0.285 cancor       yes  
#>  4 su    age     0.402 cancor       yes  
#>  5 rbc   age     0.288 cancor       yes  
#>  6 pc    age     0.239 cancor       yes  
#>  7 pcc   age     0.238 cancor       yes  
#>  8 ba    age     0.109 cancor       yes  
#>  9 bgr   age     0.125 pearson      yes  
#> 10 bu    age    -0.218 pearson      yes  
#> # ... with 1,094 more rows
\end{Soutput}
\end{Schunk}

By default, the function \texttt{calc\_assoc} calculates the association
measures for all the variable pairs at different levels of the grouping
variable and the pairwise association measures for the ungrouped data
(\texttt{overall}) when used with the \texttt{by} argument. This
behavior can be changed by setting \texttt{include.overall} argument to
\texttt{FALSE}.

\begin{Schunk}
\begin{Sinput}
ckd_assoc_by <- calc_assoc_by(df,by = "cad",include.overall = FALSE)
\end{Sinput}
\end{Schunk}

The tibble output in the conditional setting has a similar structure as
\texttt{calc\_assoc} used with no \texttt{by} argument. When used with
the \texttt{by} argument, an additional \texttt{by} column representing
the levels of the categorical variable is added in the tibble output.
The \texttt{x} and \texttt{y} variables in the output are repeated for
every level of \texttt{by} variable. In order to have multiple
\texttt{by} variables, the function \texttt{calc\_assoc} is used
multiple times with a different \texttt{by} variable each time and then
the multiple outputs are binded row wise. For calculating multiple
measures for a specific variable type, one can use
\texttt{update\_assoc} with \texttt{calc\_assoc} and then can merge
these multiple tibble outputs.

\hypertarget{section-5-corvis-visualising-association}{%
\section{Section 5: corVis: Visualising
Association}\label{section-5-corvis-visualising-association}}

We propose novel visualisations to display association and conditional
association for every variable pair in a dataset in a single plot and
show multiple bivariate measures of association simultaneously to find
out interesting patterns. Efficient seriation techniques have also been
included to order and highlight interesting relationships. These ordered
association and conditional association displays help find interesting
patterns in the dataset.

While designing these displays we considered matrix-type, linear and
network-based layouts. A matrix-type layout simplifies the effort in
finding variables, and different measures may be displayed on the upper
and lower diagonal. Linear layouts are more space-efficient than matrix
plots, but looking for variables is more challenging. Variable pairs are
usually ordered in the linear layouts by relevance (usually difference
in measures of association or across the factor levels) ,and it is
easier to omit less relevant pairs.

The chrnonic kidney dataset providing information on early stage of
Chronic Kidney Disease(CKD) from the \citep{Dua:2019} has been used to
provide illustrative examples. Table \ref{tab:ckd} provides a brief
description of set of variables from this dataset used for analysis.

\begin{Schunk}
\begin{table}

\caption{\label{tab:ckd}Variable description of the chronic kidney dataset along with the types of variables}
\centering
\begin{tabular}[t]{lll}
\toprule
Variable & Description & VariableType\\
\midrule
age & Age in years & numerical\\
bp & Blood Pressure in mm/Hg & numerical\\
bgr & Blood Glucose Random in mgs/dl & numerical\\
bu & Blood Urea mgs/dl & numerical\\
rbcc & Red Blood Cell Count in millions/cmm & numerical\\
\addlinespace
pcv & Packed  Cell Volume & numerical\\
sod & Sodium in mEq/L & numerical\\
sc & Serum Creatinine in mgs/dl & numerical\\
al & Albumin (0,1,2,3,4,5) & ordinal\\
su & Sugar (0,1,2,3,4,5) & ordinal\\
\addlinespace
htn & Hypertension (yes,no) & nominal\\
dm & Diabetes Mellitus (yes,no) & nominal\\
cad & Coronary Artery Disease (yes,no) & nominal\\
\bottomrule
\end{tabular}
\end{table}

\end{Schunk}

\hypertarget{association-matrix-plot}{%
\subsection{Association Matrix plot}\label{association-matrix-plot}}

The function \texttt{association\_heatmap} is used to display a matrix
layout with association for variable pairs in the dataset. The display
is similar to existing correlation matrix plots but with every variable
pair in the dataset. This function \texttt{association\_heatmap} takes
the calculated measures of association by \texttt{calc\_assoc} function
as input and outputs a matrix display by rendering the magnitude of
association measures with a color. The function has \texttt{lassoc} and
\texttt{uassoc} arguments for a tibble of association measures for the
lower triangle and the upper triangle of the matrix display
respectively. The \texttt{uassoc} argument is \texttt{NULL} by default
and uses the same tibble input as used by \texttt{lassoc} if not
changed. The argument \texttt{var\_order} is used for ordering or
seriating the matrix display such that highly-associated variables are
placed nearby and are easier to identify. The function also has a
\texttt{limits} argument specifying the limit of the color scale.

Figure \ref{fig:assoc-heatmap} shows an example of the association
matrix display for every variable pair in the \texttt{penguins} dataset.
The cells along the diagonal of the matrix display show the variables
present in the dataset. Every off diagonal cell is colored using a
divergent color scale with limits \([-1,1]\) representing the value of
association measure between two variables. The plot shows Pearson's
correlation for the numeric pair of variables and canonical correlation
for mixed and nominal variable pairs.

\begin{Schunk}
\begin{Sinput}
ckd <- RWeka::read.arff("../data/chronic_kidney_disease.arff")

vars <- c("age","bp","bgr","bu","rbcc","pcv","sod","sc","al","su","htn","dm","cad")
df <- ckd[,vars]


assoc <- calc_assoc(df)
association_heatmap(assoc)
\end{Sinput}
\begin{figure}

{\centering \includegraphics{rj_paper_files/figure-latex/assoc-heatmap-1} 

}

\caption[Association matrix display for penguins data showing Pearson's correlation for numeric variable pairs, canonical correlation for mixed variable pairs and categorical variable pairs]{Association matrix display for penguins data showing Pearson's correlation for numeric variable pairs, canonical correlation for mixed variable pairs and categorical variable pairs.}\label{fig:assoc-heatmap}
\end{figure}
\end{Schunk}

It shows a high positive Pearson's correlation among
\texttt{flipper\_length} and \texttt{body\_mass},
\texttt{flipper\_length} and \texttt{bill\_length}, and
\texttt{bill\_length} and \texttt{bodymass}. There seems to be a strong
negative Pearson's correlation between \texttt{flipper\_length} and
\texttt{bill\_depth}, and \texttt{bill\_depth} and \texttt{body\_mass}.

The plot also shows that there is a low canonical correlation between
\texttt{species} and \texttt{sex}, and \texttt{island} and \texttt{sex}
of the penguins suggesting low association, which traditional
correlation matrix display would omit as they are limited to numeric
variable pairs only. There is a high canonical correlation between
\texttt{island} and \texttt{species} suggesting a stronger association
among these two variables.

The high canonical correlation between \texttt{bill\_length} and
\texttt{species}, \texttt{body\_mass} and \texttt{species}, and
\texttt{flipper\_length} and \texttt{species} suggests that the body
measurements of a penguin depends on their \texttt{species}. The
variables in the display are ordered using average linkage clustering
method to find out highly associated variables quickly.

\hypertarget{multiple-association-measures-plot}{%
\subsection{Multiple Association Measures
Plot}\label{multiple-association-measures-plot}}

We also calculate multiple association measures for all the variable
pairs in the dataset and compare them. This helps in identifying pairs
of variables with a high difference among different measures or any
unusual variable pairs which should be investigated further in more
detail.

The function \texttt{pairwise\_1d\_compare} is used to display a linear
layout for comparing multiple association measures calculated for
variable pairs in the dataset. \texttt{pairwise\_1d\_compare} takes a
dataset and a set of association measures as input and outputs a heatmap
display by plotting all the variable pairs in the dataset on the Y-axis
and the association measures on the X-axis and encoding the magnitude of
association measures with a color.

The \texttt{df} argument of the function \texttt{pairwise\_1d\_compare}
is used for a dataset. The \texttt{measures} argument is for the
multiple association measures to be calculated for pairs of variables of
\texttt{df}. The default measures calculated by the function are
Pearson's, Spearman's and Kendall's correlation coefficient, canconical
correlation and normalized mutual information. The function also come up
with a \texttt{var\_order} argument which is set to \texttt{max\_diff}
by default and orders the variable pairs on the basis of maximum
difference among the magnitude of measures.

Figure \ref{fig:compare-linear} shows a linear layout comparing multiple
association measures for all the variable pairs in the penguins data.
Linear layouts are more suitable when comparing high number of
association measures.

\begin{Schunk}
\begin{Sinput}
pairwise_1d_compare(df,measures=c("pearson","spearman","kendall","cancor","nmi"))
\end{Sinput}
\begin{figure}

{\centering \includegraphics{rj_paper_files/figure-latex/compare-linear-1} 

}

\caption[Comparing multiple association measures using a linear layout]{Comparing multiple association measures using a linear layout. The display has variable pairs on the Y-axis and association measures on the X-axis. The cell corresponding to a variable pair and an association measure has been colored grey showing that the measure is not defined for corresponding pair.}\label{fig:compare-linear}
\end{figure}
\end{Schunk}

\hypertarget{conditional-association-plot}{%
\subsection{Conditional Association
Plot}\label{conditional-association-plot}}

The function \texttt{pairwise\_2d\_plot} is used to display a matrix
layout of the conditional association for variable pairs in the dataset.
The display is produced by splitting the data by a partitioning variable
and calculating association for the variable pairs at each level of
partitioning variable using \texttt{calc\_assoc} function with
conditioning variable as the \texttt{by} argument. The calculated
association measures are then displayed using bars in a matrix plot. The
height and color of the bars are coded with the value of association
measure and the level of the partitioning variable respectively. These
displays are efficient for discovering variable pair with high
differences among the levels of partitioning variable in the data.

The measures of association calculated for every variable pair at every
level of conditioning serve as input to the \texttt{pairwise\_2d\_plot}
function. The \texttt{lassoc} and \texttt{uassoc} arguments look for a
tibble of association measures for the lower triangle and the upper
triangle of the matrix display respectively. The \texttt{uassoc}
argument is set to \texttt{NULL} by default and uses the same tibble
input as used by \texttt{lassoc} if not updated. The argument
\texttt{group\_var} is responsible for the grouping variable when
plotting the bars. The default value \texttt{by} uses the conditioning
variable and \texttt{measure\_type} is useful for displaying bars with
height and color of the bars coded by the value of association measure
and the type of association measure respectively.

Figure \ref{fig:cond-assoc} shows a conditional association plot for the
\texttt{penguins} data. Each cell corresponding to a variable pair shows
three bars which correspond to the association measure (Pearson's
correlation for numeric pair and Normalized mutual information for other
combination of variables) calculated at the levels of conditioning
variable \texttt{island}. The dashed line represents the overall
association measure. The plot shows that there is a high value for
normalised mutual information between bill\_length\_mm and species for
the penguins which lived in \texttt{Biscoe} island compared to the
penguins which lived in \texttt{Dream} island. It can also be seen that
the cell corresponding to variable pair flipper\_length\_mm and
bill\_depth\_mm has a high negative overall Pearson's correlation and
for the penguins which lived in \texttt{Biscoe} island but positive
correlation for penguins which lived in \texttt{Dream} and
\texttt{Torgersen} island. This is an instance of Simpson's paradox
which can be taken into account during the modeling step.

\begin{Schunk}
\begin{Sinput}
updated_assoc <- update_assoc(num_pair = "tbl_cor",
                              mixed_pair = "tbl_nmi",
                              other_pair = "tbl_nmi")

cond_assoc <- calc_assoc_by(df, by="cad")
pairwise_2d_plot(cond_assoc)
\end{Sinput}
\begin{figure}

{\centering \includegraphics{rj_paper_files/figure-latex/cond-assoc-1} 

}

\caption[Conditional Association plot for penguins data showing Pearson's correlation for numeric pairs and normalised mutual information for categorical or mixed pairs]{Conditional Association plot for penguins data showing Pearson's correlation for numeric pairs and normalised mutual information for categorical or mixed pairs. The bars in each cell represent the value for asssociation measure colored by the conditioning variable `island`. The dashed line in each cell represents overall value of the association measure.}\label{fig:cond-assoc}
\end{figure}
\end{Schunk}

We also use linear layouts for displaying conditional association in the
package. The function \texttt{pairwise\_1d\_plot} is used for displaying
a linear layout of the conditional association for variable pairs in the
dataset. The association measures are calculated for every variable pair
at each level of partitioning variable using \texttt{calc\_assoc}
function with conditioning variable as the \texttt{by} argument.

The measures are then displayed using a dotplot where color of the dots
are coded by the level of the partitioning variable and the variable
pairs are ordered by absolute maximum value of association measure for
each of the pair of variable. These displays are also efficient for
discovering differences among the levels of partitioning variable in the
data. With the linear layouts it is easier to omit less relevant pairs
of variables by filtering the variables pairs having a higher value for
association measures than a threshold.

The measures of association calculated for every variable pair at every
level of conditioning serve as input to the \texttt{pairwise\_1d\_plot}
function. The \texttt{assoc} argument uses a tibble of association
measures calculated using \texttt{calc\_assoc} function with a
\texttt{by} variable. The argument \texttt{group\_var} is responsible
for the grouping variable when plotting the dots. The default value
\texttt{by} uses the conditioning variable and \texttt{measure\_type} is
useful for displaying dots with color of the dots coded by the type of
association measure. The \texttt{var\_order} argument is responsible for
the ordering of variable pairs in the display. If set to
\texttt{default} variable pairs are ordered alphabetically and are
ordered by absolute maximum value of association measure for every
variable pair when set to \texttt{max\_diff}.

Figure \ref{fig:linear-cond-assoc} shows a funnel-like linear display
for conditional association measures with all the variable pairs on the
y-axis, the value of association measure on x-axis and color of the
points representing the level of the grouping variable. The linear
layout becomes more useful over the matrix layout when the number of
variables and number of levels of grouping variable are high.

\begin{Schunk}
\begin{Sinput}
pairwise_1d_plot(cond_assoc)
\end{Sinput}
\begin{figure}

{\centering \includegraphics{rj_paper_files/figure-latex/linear-cond-assoc-1} 

}

\caption[Conditional Association plot using linear layout.The display has variable pairs on the Y-axis and the value of association measures on the X-axis]{Conditional Association plot using linear layout.The display has variable pairs on the Y-axis and the value of association measures on the X-axis. The points corresponding to every variable pair represents the value of association measure for different levels of the conditioning variable and the overall value of association measure.}\label{fig:linear-cond-assoc}
\end{figure}
\end{Schunk}

The \texttt{pairwise\_2d\_plot} function is also useful for comparing
various measures using the matrix layout. It plots multiple measures
among the variable pairs as bars, where each bar represents one measure
of association. Figure \ref{fig:compare-matrix} shows a matrix layout
comparing Pearson's and Spearman's correlation coefficient for the
numeric variable pairs in \texttt{penguins} data. The plot shows that
the value for both the correlation coefficients are very high for
\texttt{bill\_length} and \texttt{flipper\_length},
\texttt{bill\_length} and \texttt{body\_mass}, and
\texttt{flipper\_length} and \texttt{body\_mass} suggesting a strong
linear and montonic relationship among these variable pairs in the
dataset.

\begin{Schunk}
\begin{Sinput}
df_num <- select(df,where(is.numeric))
pearson <- calc_assoc(df_num)

spearman_assoc <- update_assoc(num_pair = "tbl_cor",
                               num_pair_argList= "spearman",
                               mixed_pair = "tbl_cancor",
                               other_pair = "tbl_nmi")
spearman <- calc_assoc(df_num, types=spearman_assoc)
compare <- rbind(pearson,spearman)
pairwise_2d_plot(compare, group_var = "measure_type")
\end{Sinput}
\begin{figure}

{\centering \includegraphics{rj_paper_files/figure-latex/compare-matrix-1} 

}

\caption[Matrix display comparing Pearson's and Spearman's correlation coefficient]{Matrix display comparing Pearson's and Spearman's correlation coefficient. All the variable pairs have similar values for both correlations.}\label{fig:compare-matrix}
\end{figure}
\end{Schunk}

\hypertarget{section-5-discussion}{%
\section{Section 5: Discussion}\label{section-5-discussion}}

We use multiple association measures in a single display for different
variable pairs which serves as a comparison tool while exploring
association in a dataset and assist in identifying unusual variable
pairs. These multiple measures can be displayed in a scatterplot matrix
similar to what \citet{tukey1985computer} proposed. They suggested that
scatterplot matrix of the scagnostics measures, which are measures
summarizing a scatterplot, can be used to identify unusual scatterplots
or variable pairs. \citet{wilkinson2005graph} used this idea with their
graph-theoretic scagnostic measures to highlight unusual scatterplots.
Similarly, \citet{kuhn2013applied} have used this idea in a predictive
modeling context. They have produced a scatterplot matrix of the
measures between the response and continuous predictors such as
Pearson's correlation coefficient, pseudo-\(R^2\) from the locally
weighted regression model, MIC and Spearman's rank correlation
coefficient to explore the predictor importance during feature selection
step. These displays show the importance of comparing multiple
association measures at once for different variable pairs.

\bibliography{RJreferences.bib}

\address{%
Amit Chinwan\\
Maynooth University\\%
Hamilton Institute\\ Maynooth, Ireland\\
%
%
%
\href{mailto:amit.chinwan.2019@mumail.ie}{\nolinkurl{amit.chinwan.2019@mumail.ie}}%
}

\address{%
Catherine Hurley\\
Maynooth University\\%
Department of Mathematics and Statistics\\ Maynooth, Ireland\\
%
%
%
\href{mailto:catherine.hurley@mu.ie}{\nolinkurl{catherine.hurley@mu.ie}}%
}
